\subsection{Estimation of roll damping from roll decay tests}
\label{se:experimental_estimation}
The roll decay test has the benefit that both roll damping and the natural frequency $\omega_0$ can be observed, but it has the drawback that roll damping at only this one frequency can be obtained. In order to extract roll damping parameters from the roll decay tests, parameters in the cubic, quadratic, or linear roll decay models should be identified. The roll angle is measured during the roll decay tests. The system identification is defined as finding the parameters that produce a simulated roll signal that best fits the roll decay test measurement. 

Two different solution approaches have been investigated for the system identification, i.e., the ``derivation approach'' (referred to as PIT in \parencite{imo_1200_2006}) and the ``integration approach'' which is similar to what \parencite{soder_assessment_2019} used. In the derivation approach the first and second roll time derivatives are calculated numerically so that the parameters in the models are the only unknowns. A least squares fit could be used on the roll motion equation to identify any parameter, including nonlinear or frequency parameter. In the integration approach, the parameters are found by solving a nonlinear problem using the least-square method. This approach requires that an ordinary differential equation to be solved for many estimated sets of parameters until the solution converges.

It should be noted that even though the approach could well handle roll equations with higher order of non-linearities in the damping term as well as a non-linear restoring term, the limited amplitudes at which the roll decay tests were conducted cannot motivate advantages of higher order models. A validation of the developed parameter identification method has been conducted by checking that parameters from simulated signals with the linear, quadratic and cubic model  (where the parameters are already known) can be identified correctly. 
The goodness of fit is described using the coefficient of determination:
\begin{equation} \label{eq:R2}
%R^2=1-\frac{SS_{res}}{SS_{tot}}
R^2=1-\frac{\sum\limit_{i=1}^{n}(y_i-\hat{y}_i)^2}{\sum\limit_{i=1}^{n}(y_i-\bar{y})^2}
\end{equation}
where $y_i$, $\bar y$, $\hat{y}_i$ represents the motion angle $\phi$, mean of $\phi$ from the model tests, and estimated $\phi$ by the system identification method, while they represent the damping coefficients in  Section 3.      



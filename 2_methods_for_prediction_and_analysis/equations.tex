\subsection{Equations}
\label{se:equations}
The roll motion can be written as \cite{himeno_prediction_1981}:
\input{equations/roll_equation_himeno}

The equation express the roll moment [Nm] along a longitudinal axis though the centre of gravity.
Where $A_{44}$ is the virtual mass moment of inertia, $B_{44}$ is the roll damping moment and $C_{44}$ is the restoring moment. $M_{44}$ represents the external moment (usually moment from external waves).

The roll damping moment $B_{44}$ is the primary interest in this paper. The $B_{44}$ is determined using model scale roll decay tests. $B_{44}$ is determined using system identification, by finding the best fit to the following equation:
\begin{equation}
A_{44} \ddot{\phi} + \operatorname{B_{44}}\left(\dot{\phi}\right) + \operatorname{C_{44}}\left(\phi\right) = 0
\end{equation}

The external moment is zero during a roll decay test, since there are no external forces present.

The $B_{44}$ can be expressed as a series expansion:  
$ B_{44} = B_1\cdot\dot{\phi} + B_2\cdot\dot{\phi}\left|\dot{\phi}\right| + B_3\cdot\dot{\phi}^3 + ...$

Truncating this series at the cubic term gives a "cubic model":
\input{equations/roll_decay_equation_cubic}

Truncating this series at the quadratic term gives a "quadratic damping model":Truncating this series at the quadratic term gives a "quadratic model":
\begin{equation}
A_{44} \ddot{\phi} + \left(B_{1} + B_{2} \left|{\dot{\phi}}\right|\right) \dot{\phi} + \operatorname{C_{44}}\left(\phi\right) = 0
\end{equation}


Truncating this series at the linear term gives a "linear model":
\begin{equation}
A_{44} \ddot{\phi} + B_{1} \dot{\phi} + \operatorname{C_{44}}\left(\phi\right) = 0
\end{equation}


The linear model \cite{henry_peter_piehl_ship_nodate} can be solved analytically, where the natural frequency of the motion is obtained by:
\begin{equation}
\omega_{0} = \sqrt{\frac{C}{A_{44}}}
\end{equation}



\subsection{Ikeda's method: from strip-theory to semi-empirical formulas}
\label{se:semi-empirical methods}
Ikeda's method divides roll damping into five damping components: the friction component $B_F$, the eddy component $B_E$, the lift component $B_L$, the wave component $B_W$ and the bilge keel component $B_{BK}$, as in the following Eq.(\ref{eq:ikeda}), 
\begin{equation} \label{eq:ikeda}
B_{44} = B_F + B_E + B_L + B_W + B_{BK}
\end{equation}
where the wave and eddy components require strip-theory based hydrodynamic analysis to obtain the ship's shape coefficients. The hydrodynamic analysis requires the ship's exact hull geometries. It might be time consuming to build the geometry model and perform the strip-theory based hydrodynamic analysis. Sometimes, a ship's hull geometry is simply not available for such purposes. 

A simplified Ikeda method (SI-method) was proposed by \parencite{kawahara_simple_2011} and is used in this study to calculate all the damping components including the eddy component $B_E$ and wave component $B_W$. The semi-empirical formulas describe four of the five roll damping components at motion frequency $\omega$ for a given roll amplitude $\phi_a$ at zero ship speed. A speed dependency was introduced by adding a fifth damping term $B_L$ and a speed correction to $B_W$ and $B_E$ as described in \parencite{ikeda_velocity_1979}, giving a function: 

\begin{equation} \label{eq:simplified_ikeda_equation}
\left( B_{44}, \  B_{F}, \  B_{W}, \  B_{E}, \  B_{BK}, \  B_{L}\right) = \operatorname{Ikeda_{simplified}}\left(L_{pp},beam,C_{b},A_{0},OG,\phi_{a},BK_{L},BK_{B},\omega,T,V\right)
\end{equation}


The formulas within $f$ can be referred to \parencite{ikeda_velocity_1979, kawahara_simple_2011} with the implementation in the paper by \parencite{alexandersson_martinlarsalbertrolldecay-estimators_2020}.
It should be noted that this method may be only efficiently used to estimate the roll damping of ships within the boundaries \parencite{kawahara_simple_2011}:
\begin{equation}
    \label{eq:SI_limits}
     \left\{
     \begin{array}{ll}
    0.5 \leq C_b \leq 0.85,\hspace{0.5cm} 
    0 \leq \hat{\omega} \leq 1.0,
    \hspace{0.5cm}
    0.9 \leq A_0 \leq 0.99,\\
    2.5\leq Beam/T \leq 4.5, \hspace{1cm}
    0.01 \leq BK_B/Beam \leq 0.06, \\
        -1.5 \leq OG/T \leq 0.2,
     \hspace{1cm}
    0.05 \leq BK_L/L_{PP} \leq 0.4.
    \end{array}
    \right.
\end{equation}


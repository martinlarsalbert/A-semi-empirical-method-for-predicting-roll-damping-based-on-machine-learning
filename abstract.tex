{\footnotesize
\noindent\rule{\columnwidth}{0.4pt}
\section*{Abstract}\label{se:abstract}
IMO has in the second generation intact stability criteria addressed the importance of ships having sufficient roll damping to avoid parametric roll and large roll motions in dead ship condition as well as excessive acceleration. 
Semi-empirical methods such as Ikeda’s method is widely used to predict roll damping for practical purposes, especially in the early design stage of ships where computational fluid dynamics or experimental model tests are not feasible options. Recent work has shown that the applicability and accuracy of Ikeda’s method for modern hull forms are somewhat uncertain which is very unfortunate, especially for analysis of parametric roll. A small error in the roll damping prediction can make the difference between having no parametric roll and disaster.

This paper presents a new semi-empirical method as a complement to the simplified Ikeda's method. The new method has been developed using a machine learning based approach on a roll damping database built by performing parameter identification of roll decay model tests. 
The new method is estimated to have higher accuracy than the simplified Ikeda's method which has been investigated using cross validation on the roll damping database. The paper also suggests a lower limit for the ship draughts where the simplified Ikeda's method can be used.


\noindent {\scriptsize \emph{Keywords}: Roll damping; Roll decay; Ikeda’s method; Simplified Ikeda’s method; Ship motions; Machine Learning}
}
\newline
\noindent\rule{\columnwidth}{0.4pt}
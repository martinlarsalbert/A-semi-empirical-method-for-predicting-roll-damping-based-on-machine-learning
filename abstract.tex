{\footnotesize
\noindent\rule{\columnwidth}{0.4pt}
\section*{Abstract}\label{se:abstract}

Semi-empirical methods such as the Ikeda’s method or simplified Ikeda’s method are widely used to predict roll damping for practical purposes. Recent work has shown that the applicability and accuracy of Ikeda’s method for modern hull forms are somewhat uncertain which is very unfortunate, especially for analysis of parametric roll. 
%A small error in the roll damping prediction can make the difference between having no parametric roll and disaster.
This paper investigates the accuracy of the simplified Ikeda's method and proposes some correction factors that can improve the accuracy for modern ships. The proposed corrections have been developed using a regression based approach on a roll damping database built by performing parameter identification of roll decay model tests. The simplified Ikeda's is estimated to have higher accuracy when the new corrections are applied which has been investigated using cross validation on the roll damping database. 


\noindent {\scriptsize \emph{Keywords}: Roll damping; Roll decay; Ikeda’s method; Simplified Ikeda’s method; Ship motions; Parameter identification technique.}
}
\newline
\noindent\rule{\columnwidth}{0.4pt}
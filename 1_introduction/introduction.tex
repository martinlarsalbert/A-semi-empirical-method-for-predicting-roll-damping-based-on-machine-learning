\section{Introduction}
\label{se:introduction}

In the second generation of intact stability criteria, the IMO has addressed the importance of ships having sufficient roll damping to avoid parametric roll and large roll motions in dead ship condition as well as excessive acceleration \parencite{imo_finalization_2016}. 
Since the scale effect of the damping is mainly associated with the skin friction on ship hulls and the friction only contributes very little to a full scale ship's total roll damping, experimental tests is a widely accepted method to estimate a ship's roll damping \parencite{imo_1200_2006}. A lot of research activities have been putting on methods to estimate roll damping from experimental tests. For example, Hua et al. (2011) investigated a second order oscillator model to approximate the a ship's roll damping based on both experiments at different scales and numerical calculations. \parencite{soder_ikeda_2019} presented unique experimental set-ups in model scale and full-scale to evaluate roll damping properties from the test results. 
As the rapid increase of computation capability, the CFD methods have also been used to calculate the roll damping as in \parencite{kristiansen_experimental_2014} and \parencite{henry_peter_piehl_ship_2016}.  

However, in the early stage of ship design, when only limited information is available, such as the ship's principal dimensions and the basic hull geometry, computational fluid dynamics or experimental model tests are not feasible options.
Therefore, simpler methods are widely used. Simpler strip theory methods can give quite accurate prediction of a ship's pitch, heave, sway and yaw motions, but often overestimates the roll motions because of significant viscous effects on the roll damping \parencite{kawahara_simple_2011}. Therefore other methods are used to obtain the roll damping at early design stage. Several semi-empirical methods were proposed in the late 70th as described by  \parencite{himeno_prediction_1981}. The most recognize method was developed in a series of research articles \parencite{ikeda_roll_1978,ikeda_eddy_1978,ikeda_roll_1979,ikeda_components_1978,ikeda_velocity_1979}, often referred to as the Ikeda's method. This method divides the roll damping into five damping components, i.e., the friction component, the eddy component, the lift component, the wave component and the bilge keel component. This semi-empirical method is also recommended by \parencite{ittc_ittc_2011}. As reported by  \parencite{kawahara_simple_2011} and \parencite{soder_ikeda_2019}, Ikeda's method may not work well for unconventional ships, such as container ship and PCTC with pronounced flare bow and stern structures. Concerning that the simplified Ikeda’s
method may considerably underestimate the eddy making component
of damping of full hull forms, Rudakovic and Bačkalov (2017) adjusted the simplified Ikeda’s method and  extended its application to stability analysis of inland ships.
In addition, the rather widely used simplified Ikeda's method \parencite{kawahara_simple_2011} was constructed by a series of old ship hulls, while it might not give a fare estimation of roll damping of today's much optimized ship hull forms. 

The objective of this paper is to propose a new semi-empirical method as a complement to the simplified Ikeda's method. First, the accuracy of the simplified Ikeda's method is investigated in order to identify weaknesses and strengths to see which parts of the simplified Ikeda's method that can be reused in the new method.  
Then a semi-empirical model is developed using a regression based approach on a large database of roll decay model tests composed of more than 250 ships at SSPA.
To make the completeness of the paper section \ref{se:methods_for_prediction_and_analysis} presents the basic governing equations of roll motions and how roll damping can be obtained from roll decay tests or the simplified Ikeda's method. 
The roll decay test database is presented in section \ref{se:database_of_roll_decay_tests}, as well as the roll damping estimated by the system identification techniques as damping references, which are compared with the simplified Ikeda's method to identify weaknesses and strengths of this method. Section \ref{se:regression} proposes a new regression-based method to improve the accuracy of roll damping in comparison with simplified Ikeda's method. Conclusions are given in section \ref{se:conclusions}.  

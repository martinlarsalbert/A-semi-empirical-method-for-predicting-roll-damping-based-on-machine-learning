\section{Introduction}
\label{se:introduction}

In the second generation of intact stability criteria, the IMO addressed the importance of ships having sufficient roll damping to avoid large roll motions and parametric rolling, as well as excessive acceleration \parencite{imo_finalization_2016}. The phenomena have received more attention since the late 90’s. Modern ship designs (especially volume carriers) seem to be more vulnerable to the phenomena as was seen in the APL China casualty. The damping of roll motion plays an important part during the above mentioned phenomena and \parencite{soder_ikeda_2019} showed that the relatively small difference in the roll damping prediction they got using different methods could mean the difference between sever roll angles and hardly noticeable motions. 

Experimental model tests are a widely accepted method to estimate a ship's roll damping since the scale effect of the damping is mainly associated with the skin friction on ship hulls and the friction only contributes very little to a full-scale ship's total roll damping\parencite{imo_1200_2006}. As the rapid increase of computation capability, the computational fluid dynamics (CFD) methods have also been used to calculate the roll damping as in \parencite{kristiansen_experimental_2014} and \parencite{henry_peter_piehl_ship_2016}.  
However, in the early stage of ship design, when only limited information is available, such as the ship's principal dimensions and the basic hull geometry, both CFD methods and experimental model tests are sometimes not attractive options. Therefore, simpler methods are widely used. 
%Simple strip theory methods can give quite accurate prediction of a ship's pitch, heave, sway and yaw motions, but often overestimates the roll motions because of significant viscous effects on the roll damping \parencite{kawahara_simple_2011}. Therefore other methods are used to obtain the roll damping at early design stage. 
Several semi-empirical methods were proposed in the late 70s \parencite{himeno_prediction_1981}. The most recognized method was developed in a series of research articles \parencite{ikeda_roll_1978,ikeda_eddy_1978,ikeda_roll_1979,ikeda_components_1978,ikeda_velocity_1979}, often referred to as the Ikeda's method based on strip theory based analysis. This semi-empirical method is also recommended by \parencite{ittc_ittc_2011}. 
There also exist a newer and simplified version of the Ikeda's method \parencite{kawahara_simple_2011} (named as SI-method here) where (unlike Ikeda original method) strip theory calculations are not needed, which makes it much easier to use in design stages of ships. This method was developed as a regression on calculation results from the Ikeda's method for a series of parameterized hull shapes and is claimed to have almost the same accuracy as the original method. As reported by  \parencite{kawahara_simple_2011} and \parencite{soder_ikeda_2019}, Ikeda's method may not work well for unconventional ships, such as container ship and PCTC with pronounced flare bow and stern structures. 

The main objective of this paper is to investigate the accuracy of the SI-method for modern ships using more than 250 roll decay tests conducted at SSPA during the past 15 years. Furthermore, possible ways to improve the accuracy of the SI-method for modern ships will be investigated.
To make the completeness of the paper, Section \ref{se:methods_for_prediction_and_analysis} presents the basic governing equations of roll motions and how roll damping can be obtained from roll decay tests and the SI-method. 
Based on the roll decay test database, different methods to estimate roll damping are compared in Section \ref{se:accuracy_SI_method}. Section \ref{se:correction_SI_method} proposes two new regression-based methods to improve the accuracy of roll damping in comparison with the SI-method. Conclusions are given in section \ref{se:conclusions}.  

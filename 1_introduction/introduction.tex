\section{Introduction}
\label{se:introduction}

%\begin{itemize}
%    \item Roll damping important
%    \item Roll decay test at zero speed and %speed.
%    
%\end{itemize}
			
Ship roll has been the subject of several studies over the past decades. Roll damping devices include tuned liquid dampers such as anti-rolling tanks as well as exterior appendages such as bilge keels. Both passive and active anti-rolling devices have been investigated. Theoretical, numerical and experimental studies have been performed and presented in the literature. A classic example of an experimental study is presented by [1] for roll without forward speed. Different components of damping were studied in a series of publications by [2], [3] and [4]. Both naked hulls and hulls with bilge keels were
studied therein. Due to a significant amount of
experimental data for conventional hulls with bilge keels, for instance as presented in the above mentioned publications, good empirical formulas have existed for several decades. The report by Himeno (1981) presents a comprehensive summary of the state-of-the-art at that time, and formulas presented therein are still used today IMO (2016).

The computer power and availability of fast, stable NaviérStokes solvers have made it feasible to a larger degree to study the problem numerically in recent years, and this trend is expected to continue. However, having model tests as benchmark data will remain a requirement. The experiemental methods and semi-empirical formulas are still in a great demand espeically in the ship conceptual design and inspections, see IMO (2016). In this paper we present results from experiments and semi-empirical method proposed by Ikeda\'s method of a big database towing tank tests from SSPA. In order to faciliate the usage of the SSPA roll-decay test base, the Ikeda\'s method is further developed to reflect the state-of-the-art ship design with correct roll damping properties. 


\section{Introduction}
\label{se:introduction}
%If an undamaged ship capsizes, it will most likely do so by reaching a critical roll angle, for obvious geometrical reasons. Ship stability in roll %has therefore a lot of attention when designing ships. When also considering the dynamics of ships, the damping of the roll motion will also be very %important, in order to avoid large motions when near the resonance frequency. Verifying that a ship has sufficient roll damping is therefore a %critical part of all ship designs.

\subsection{Thesis statement}
A large set of relatively new empirical data of model roll decay tests is used to investigate the accuracy of Ikedas simplified method. The data is also used to create a new semi empirical method to predict ship roll damping.

\subsection{Purpose of research paper}
Many crucial decisions that are made during the early design stage of ships, are based on predictions where the roll damping has been calculated with semi-empirical methods. The accuracy of these methods, when applied to modern ship designs, is however quite uncertain. Söder et al. \cite{soder_ikeda_2019} has for instance investigated Ikeda’s method for modern volume carriers where it was concluded that the speed dependence of the bilge keels damping was underestimated.


%\subsection{Research Area}
%IMO has in the second generation intact stability criteria \cite{imo_second_nodate} addressed six possible dynamic stability failures. Having %sufficient roll damping is very important in these failure modes, especially in the modes: \emph{Parametric roll}, \emph{Dead Ship condition} and %\emph{Excessive acceleration}. Semi-empirical formulas to predict roll damping are examined in this project using \emph{Machine Learning} (ML) %techniques. Historical data from roll decay scale model tests is used to examine the accuracy of existing semi-empirical methods to make better %predictions of roll damping for future ships.


%\subsection{Methods}
%The accuracy of state of the art semi-empirical roll damping methods are investigated by benchmarks of a large number \textcolor{red}{(how many?)} of %roll decay model tests. Machine Learning techniques are used to investigate if the best semi-empirical methods can be improved. The methodology used %in this project is general, so that it can, with some effort, be reused to other aspects of ship dynamics such as: seakeeping and manoeuvring.

\section{Background}
\label{se:background}

\subsection{Equations}
\label{se:equations}

The roll motion can be written as \cite{himeno_prediction_1981}:
\begin{equation} \label{eq:roll_equation_himeno}
A_{44} \ddot{\phi} + \operatorname{B_{44}}\left(\dot{\phi}\right) + \operatorname{C_{44}}\left(\phi\right) = \operatorname{M_{44}}\left(\omega t\right)
\end{equation}

The equation expresses the roll moment [Nm] along a longitudinal axis though the centre of gravity.
Where $A_{44}$ is the virtual mass moment of intertia, $B_{44}$ is the roll damping moment and $C_{44}$ is the restoring moment. $M_{44}$ represents the external moment (usually moment from external waves).
The roll damping moment $B_{44}$ is the primary interest in this paper. The $B_{44}$ is determined using model scale roll decay tests. $B_{44}$ is determined using system identification, by finding the best fit to the following equation:
\begin{equation} \label{eq:roll_decay_equation_general_himeno}
A_{44} \ddot{\phi} + \operatorname{B_{44}}\left(\dot{\phi}\right) + \operatorname{C_{44}}\phi = 0
\end{equation}

The external moment is zero during a roll decay test, since there are no external forces present.
The $B_{44}$ can be expressed as a series expansion:
\begin{equation}
\label{eq:dampingSeriesExansion}
B_{44} = B_1\cdot\dot{\phi} + B_2\cdot\dot{\phi}\left|\dot{\phi}\right| + B_3\cdot\dot{\phi}^3 + ...$
\end{equation}
Truncating this series at the quadratic term gives a "quadratic damping model":
\begin{equation} \label{eq:b44_quadratic_equation}
\operatorname{B_{44}}\left(\dot{\phi}\right) = B_{1} \dot{\phi} + B_{2} \left|{\dot{\phi}}\right| \dot{\phi}
\end{equation}

Assuming quadratic damping the roll decay equation is written:
\begin{equation} \label{eq:roll_decay_equation_himeno_quadratic}
A_{44} \ddot{\phi} + \left(B_{1} + B_{2} \left|{\dot{\phi}}\right|\right) \dot{\phi} + \operatorname{C_{2}} \phi = 0
\end{equation}

The restoring moment can be expressed using the $GZ$ curve:
\begin{equation} \label{eq:restoring_equation}
\operatorname{C_{44}}\left(\phi\right) = g m \operatorname{GZ}\left(\phi\right)
\end{equation}

where $m$ is the mass of the ship
It is common that the restoring moment is linearized using the meta centric height $GM$:
\begin{equation} \label{eq:restoring_equation_linear}
\operatorname{C_{44}}\left(\phi\right) = GM g m \phi
\end{equation}

Introducing a helper coefficient $C$: 
\begin{equation} \label{eq:C_equation}
C = \frac{\operatorname{C_{44}}\left(\phi\right)}{\phi}
\end{equation}

In the case of linearized restoring moment $C$ can be written as:
\begin{equation} \label{eq:C_equation_linear}
C = GM g m
\end{equation}

\begin{equation} \label{eq:roll_decay_equation_himeno_quadratic_c}
A_{44} \ddot{\phi} + C \phi + \left(B_{1} + B_{2} \left|{\dot{\phi}}\right|\right) \dot{\phi} = 0
\end{equation}

It is common to rewrite the roll decay equation by dividing with $A_{44}$ and conducting the following substitutions:
\begin{equation} \label{eq:zeta_equation}
2 \omega_{0} \zeta = \frac{B_{1}}{A_{44}}
\end{equation}

\begin{equation}
d = \frac{B_{2}}{A_{44}}
\end{equation}

\begin{equation}
\omega_{0} = \sqrt{\frac{C}{A_{44}}}
\end{equation}

Commonly used quadratic roll decay equation:
\begin{equation} \label{eq:roll_decay_equation_quadratic}
\omega_{0}^{2} \phi + \left(d \left|{\dot{\phi}}\right| + 2 \omega_{0} \zeta\right) \dot{\phi} + \ddot{\phi} = 0
\end{equation}

And linear roll decay equation is obtained when $d=0$
\begin{equation}
\omega_{0}^{2} \phi + 2 \omega_{0} \zeta \dot{\phi} + \ddot{\phi} = 0
\end{equation}


			

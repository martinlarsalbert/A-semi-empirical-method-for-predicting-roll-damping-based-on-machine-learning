\section{Conclusions}
\label{se:conclusions}
A large experimental test database from SSPA including about 250 existing roll decay model tests is used in this study to improve current semi-empirical methods for roll damping prediction. First, the parameter identification technique was used to extract roll damping coefficients from these tests. The method was found to work very well to identify the parameters in the linear, quadratic and cubic mathematical models, while a quadratic damping model is sufficient to reproduce most of the roll decay tests.

It is shown that the Simplified Ikeda's method (SI-method) gives very bad prediction of roll damping in comparison with the model test results, because most of the ships in the SSPA database have main dimensions outside the limits of this method. However, the ships in the database are recognized as some kind of representation of modern merchant ships that have been tested at SSPA during the past 15 years, including for instance oil tankers, LNG-tankers, passenger ships, car carriers and others. 
%It should also noted that the selection of ships is somewhat biased based on ship types where SSPA has had a large market share. Container ships are for instance almost entirely missing from this data. But it is regardless reasonable to conclude that the SI-method cannot be used within its limitations for many modern ships. 
Furthermore, the original Ikeda method using strip theory based hydrodynamic analysis was also implemented to investigate possible errors in the SI-method. 
%This method was found to be in much more agreement with the model test results. 
It was found that wave damping component $B_W$ from the SI method was giving very high errors when the beam to draught ratio was above the limiting value of 4.5. These errors are not originating from the strip theory based Ikeda method, while must come from the extrapolation of the polynomials within the SI-method leading to large extrapolation errors. 

To predict roll damping for modern hull shapes beyond the limits of the present SI-method, two approaches were investigated how the SI-method could be used anyway, outside its limitations based on the SSPA model test database.
%The ``limited" approach seems to be the best and should be the preferred way to deal with this situation rather than the ``unlimited" approach which has large deviations. The accuracy was however quite low also when the  ``limited" approach was used.
Firstly, some correction factors were proposed to the five damping components in the SI-method using regression. Secondly, a complete new method was developed using regression on the test database. The accuracy of these methods were cross validated with significantly improved accuracy. The proposed corrections seem to improve the accuracy of the SI-method for modern ships outside the limits. The new regression model has the highest accuracy, but it is not that much higher than what was achieved with the original Ikeda method. Further research effort should be devoted to make an updated version of the SI-method, using the same approach but for a parametrization that gives a better representation of modern ship hulls. While waiting for a better method to be developed, Simplified Ikeda can be extended for modern ships using the new regression or the correction factors proposed in this paper. 

%should be the preferred way to adress the urgent task of developing a new simplified semi empirical method to predict ship roll damping for modern ships. 
%Even though a very thorough cross validation was conducted to estimate the accuracy of the regression, there is always a risk that the model is too specialized on the selection of ships that are in the database (which may be biased). The regression is therefore believed to be less general than the original Ikeda method.




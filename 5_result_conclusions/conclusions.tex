\section{Conclusions}
\label{se:conclusions}
A roll damping database has been created based on a large number of roll decay model conducted at SSPA Maritime Dynamics Laboratory during the past 15 years. The database contain model tests with modern ship hull geometries for oil tankers and LNG tankers but also various other ship types. This database was used to investigate the accuracy of simplified Ikeda's method, being the state of the art semi empirical method to predict roll damping. 
It was found that almost all ships in this database have main dimensions outside the limits of the simplified Ikeda's method, which confirms that this method cannot be used for most modern ships, at least if one should be within the input limits. Exceeding the input limits was investigated and it was found that inputs outside the limits gives large errors, especially for the wave damping component $B_W$. Using the boundary values when an input are exceeded is a much better approach and gives much better accuracy and should therefore be the preferred way to handle this situation.

A roll damping database based on model test has been compared with predictions using  the simplified Ikeda's method. The database and the predictions show agreement for some of the cases but poor agreement especially for small draft to beam ratios, which may typically occur for ballast loading condition. It seems that exceeding the limits of the the simplified Ikeda's method will give poor results. These limits are unfortunately not so well defined in \parencite{kawahara_simple_2011}. The comparison in the present paper suggest a lower limit of the draught to length ratio of $T/L_{pp}>0.034$.

A regression model has been developed which has better accuracy than the simplified method for the present roll damping data. The accuracy of the regression model has been investigated using extensive cross validation. The authors believe that the model is able to predict the roll damping at natural frequency for ships with main dimensions similar to the once in this paper within the estimated accuracy. If higher accuracy is needed, more detailed methods should be used. Compared to the simplified Ikeda's method this new method can predict the roll damping only at the natural frequency. This may still be valuable information however, even when investigating other frequencies and for many cases for instance parametric roll, roll damping at natural frequency is of primary interest.


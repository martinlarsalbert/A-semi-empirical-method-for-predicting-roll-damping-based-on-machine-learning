\section{Conclusions}
\label{se:conclusions}
A large experimental test database from SSPA including about 250 existing roll decay model tests is used in this study to improve current semi-empirical methods for roll damping prediction. First, the parameter identification technique was used to extract roll damping coefficients from these tests. The method was found to work very well in identifying the parameters in the linear, quadratic and cubic mathematical models, while a quadratic damping model is sufficient to reproduce most of the roll decay tests. It is demonstrated that predictions using the simplified Ikeda's method (SI-method) showed poor agreement with model tests outside its limits but acceptable agreement for the ships within limits. The wave-damping component $B_W$ seemed to be the main source of the error. The ships in the database are recognized as being representative of modern merchant ships that have been tested at SSPA during the past 15 years, including, for instance oil tankers, LNG-tankers, passenger ships, car carriers, and others. That so many of these ships exceed the limits, with expected poor results from the SI-method, is a bit worrying. Furthermore, the original Ikeda method using strip theory based hydrodynamic analysis was also implemented and was found to agree much better with the model tests, also outside the limits of the SI-method. It can therefore be concluded that the wave damping error is caused by extrapolation rather than by  errors in the original Ikeda method.

To predict roll damping for modern hull shapes beyond the limits of the present SI-method, two approaches were investigated as to how the SI-method could be used anyway, outside its limitations and based on the SSPA model test database. First, some correction factors were proposed to the five damping components in the SI-method using regression. Second, a completely new method was developed using regression on the test database. The accuracy of these methods were cross validated with significantly improved accuracy. The proposed corrections seem to improve the accuracy of the SI-method for modern ships outside the limits. The new regression model has the highest accuracy, but it is still lower than the SI-method within its limits or the original Ikeda method. Further research efforts should be devoted to creating an updated version of the SI-method. While waiting for a better method to be developed, simplified Ikeda can be extended for modern ships using the new regression or the correction factors proposed in this paper. 





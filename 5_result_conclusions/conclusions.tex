\section{Conclusions}
\label{se:conclusions}
The roll damping database has been compared with predictions with the implementation of the Simplified Ikeda method. The database and the predictions show agreement for some cases but poor agreement for other cases, especially for small draft to beam ratios, typically for Ballast loading condition for many modern ships. It seems that exceeding the limits of the the Simplified Ikeda method will give poor results. These limits are unfortunately not so well defined in \parencite{kawahara_simple_2011}. The comparison in the present paper suggest a lower limit of the draught to length ratio of $T/L_{pp}>0.034$.

A regression model has been developed which has better accuracy than the Simplified method for the present roll damping data. The accuracy of the regression model has been investigated using extensive cross validation. The model should be able to predict the roll damping for ships with main dimensions similar to the once in this paper within the estimated accuracy.
If higher accuracy is needed, more advanced methods can be used, such as the original Ikeda method with strip calculations, CFD or model testing, to get reliable roll damping predictions.

\section{Discussion}
The test setup in the free roll decay tests differs from the forced roll motion model tests that Ikeda \parencite{ikeda_velocity_1979} used to derive the original method. This is not believed to have any large impact as long as roll damping near the natural frequency is considered. The Ikeda model tests were also conducted with smaller models (2 meters compared to 3-6 meters). But there is a skin friction component $B_F$ to the Ikeda method that should handle this scale effect.


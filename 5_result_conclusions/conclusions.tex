\section{Conclusions}
\label{se:conclusions}
The authors of this paper have had the privilege to analyze data from about 250 existing roll decay model tests which is considered as a rather large study in this field. A parameter identification techniques (PIT) was used to extract roll damping coefficients from these tests.The PIT was found to work very well to identify the parameters in the linear, quadratic and cubic mathematical models. 
It was found that a quadratic damping model is sufficient to reproduce most of the roll decay tests.

When investigating the accuracy of the Simplified Ikeda's method (SI-method) it was found that most of the ships in the SSPA database have main dimensions outside the limits of this method. The ships in the database are considered to be some kind of representation of modern merchant ships, based on what has been tested at SSPA during the past 15 years, including for instance oil tankers, LNG-tankers, passenger ships, car carriers and others. The selection of ships is therefore somewhat biased based on ship types where SSPA has had a large market share. Container ships are for instance almost entirely missing from this data. But it is regardless reasonable to conclude that the SI-method cannot be used within its limitations for many modern ships. 

Two approached were investigated how the SI-method could be used anyway, outside its limitations. The ``limited" approach seems to be the best and should be the preferred way to deal with this situation rather than the ``unlimited" approach which has large deviations. The deviations seem to originate from the wave damping coefficient. Also the bilge keel damping coefficient seems to suffer from extrapolation when the ``unlimited" approach is used. 
Therefore, it is an urgent task to propose either a complete new method to replace the SI-method, or make corrections to the SI-method to accurately predict a modern ship’s roll damping coefficients. A preliminary attempt is presented in this paper based on the SSPA model test database.

Firstly a attempt to develop a complete new method was done using regression on the SSPA model test database. Secondly an attempt to propose corrections to the SI-method was developed. These corrections were also developed using regression. The accuracy of these methods were estimated using cross validation. The proposed corrections seem to improve the accuracy of the SI-method for modern ships outside the limits. The new regression model has the highest accuracy however, which means that a complete new method should be the preferred way to address this urgent task, rather than proposing corrections to the existing SI-method. The new regression presented in this paper is however a preliminary attempt only. Finding a new method to completely replace the SI-method in the future is considered as a much larger task, larger than the scope of this paper. 

The authors also hope that the database used in this paper can be further improved, especially by improving the meta data about the ships tested, their ship types and more detailed information about their hull geometries. 




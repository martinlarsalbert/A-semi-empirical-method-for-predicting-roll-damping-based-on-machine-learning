\section{Conclusions}
\label{se:conclusions}
The authors of this paper have had the ability to analyze data from about 250 existing roll decay model tests which is considered as a rather large study in this field. A parameter identification techniques (PIT) was used to extract roll damping coefficients from these tests.The PIT was found to work very well to identify the parameters in the linear, quadratic and cubic mathematical models. 

It was found that a quadratic damping model is sufficient to reproduce most of the roll decay tests, but that a cubic model is needed for some cases. 

When investigating the accuracy of the SI-method it was found that most ships in the SSPA database had main dimensions outside the limits of this method. The ships in the database are considered to be some kind of representation of modern merchant ships, based on what has been tested at SSPA during the past 15 years, including for instance oil tankers, LNG-tankers, passenger ships, car carriers and others. The selection of ships is therefore somewhat biased based on ship types where SSPA has a large market share. Container ships are for instance almost entirely missing from this data. But it is regardless reasonable to conclude that the SI-method cannot be used within its limitations for many modern ships.  

Even though the ships in the database is not a perfect representation of all ships  


SI-method limits
''limited'' ''unlimited''


The main objective of this paper is to investigate the accuracy of the SI-method
The second objective is to propose some corrections to this method

The conducted cross validation show that the proposed correction of the SI-method seem to improve the accuracy for modern ships. The accuracy of the New regression, which is a pure regression model is however quite much higher. 





A roll damping database has been created based on a large number of roll decay model conducted at SSPA Maritime Dynamics Laboratory during the past 15 years. The database contain model tests with modern ship hull geometries for oil tankers and LNG tankers but also various other ship types. This database was used to investigate the accuracy of simplified Ikeda's method, being the state of the art semi empirical method to predict roll damping. 
It was found that almost all ships in this database have main dimensions outside the limits of the simplified Ikeda's method, which confirms that this method cannot be used for most modern ships, at least if one should be within the input limits. Exceeding the input limits was investigated and it was found that inputs outside the limits gives large errors, especially for the wave damping component $B_W$. Using the boundary values when an input are exceeded is a much better approach and gives much better accuracy and should therefore be the preferred way to handle this situation.



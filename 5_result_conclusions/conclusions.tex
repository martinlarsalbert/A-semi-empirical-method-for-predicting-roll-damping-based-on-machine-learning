\section{Conclusions}
\label{se:conclusions}
The authors of this paper have had the privilege to analyze data from about 250 existing roll decay model tests which is considered as a rather large study in this field. A parameter identification techniques (PIT) was used to extract roll damping coefficients from these tests.The PIT was found to work very well to identify the parameters in the linear, quadratic and cubic mathematical models. 
It was found that a quadratic damping model is sufficient to reproduce most of the roll decay tests.

When investigating the accuracy of the Simplified Ikeda's method (SI-method) it was found that most of the ships in the SSPA database have main dimensions outside the limits of this method. The ships in the database are however considered to be some kind of representation of modern merchant ships. The selection of ships is based on what has been tested at SSPA during the past 15 years, including for instance oil tankers, LNG-tankers, passenger ships, car carriers and others. The selection of ships is therefore somewhat biased based on ship types where SSPA has had a large market share. Container ships are for instance almost entirely missing from this data. But it is regardless reasonable to conclude that the SI-method cannot be used within its limitations for many modern ships. 

Two approached were investigated how the SI-method could be used anyway, outside its limitations. The ``limited" approach seems to be the best and should be the preferred way to deal with this situation rather than the ``unlimited" approach which has large deviations. The accuracy was however quite low also when the  ``limited" approach was used.

In order to get a better understanding of what is causing the error in the SI-method, the original Ikeda method was also implemented. This method was found to be in much more agreement with the model test results. It was found that wave damping component $B_W$ from the SI method was giving very high errors when the beam to draught ratio was above the limiting value of 4.5. It was found that these errors are not originating from the original Ikeda method itself. The errors must come from the extrapolation (which is needed for many modern ships) of the polynomials within the SI-method, which apparently gives large extrapolation errors. It is an urgent task to come up with a better semi empirical method to predict roll damping for modern hull shapes, beyond the limits of the present SI-method. 

An attempt to develop such a method was presented in this paper based on the SSPA model test database. Firstly a attempt to develop a complete new method was done using regression on the SSPA model test database. Secondly an attempt to propose corrections to the SI-method was developed. These corrections were also developed using regression. The accuracy of these methods were estimated using cross validation. The proposed corrections seem to improve the accuracy of the SI-method for modern ships outside the limits. The new regression model has the highest accuracy however, but it is not that much higher than what was achieved with the original Ikeda method. But even though a very thorough cross validation was conducted to estimate the accuracy of the regression, there is always a risk that the model is too specialized on the selection of ships that are in the database (which may be biased). The regression is therefore believed to be less general than the original Ikeda method. It is therefore in the authors belief that making an updated version of the SI-method, using the same approach but for a parametrization that gives a better representation of modern ship hulls should be the preferred way to adress the urgent task of developing a new simplified semi empirical method to predict ship roll damping for modern ships. 



